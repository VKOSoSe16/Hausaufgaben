\documentclass[a4paper]{article}
\usepackage{amsmath}
\usepackage{amsfonts}
\usepackage{amssymb}
\usepackage[ngerman]{babel}  
\usepackage[utf8]{inputenc}

\usepackage{../gail}
\usepackage{../dadp}

\setfirstauthor{Walter Stieben}
\setfirstauthorID{4stieben@inf}
\setsecondauthor{Tim Reipschläger}
\setsecondauthorID{4reipsch@inf}
\setthirdauthor{Louis Kobras}
\setthirdauthorID{4kobras@inf}
\setfourthauthor{Hauke Stieler}
\setfourthauthorID{4stieler@inf}
\settitle{Lösungsstrategien für NP-schwere Probleme der
Kombinatorischen Optimierung}
\setsheetnumber{3}
\setstartdate{2016}{04}{11}
\setdatefreq{7}
\setsectionstyletasksalphnum{}

\begin{document}
	\maketitle
	\section{}
		\subsection{}
		\subsection{}
	\section{}
		\subsubsection{}
			Wenn $\forall v\in D: grad(v)=n>k$ gelten soll, dann gibt es zu viele Kanten ($(n)^2$ viele), als dass man diese durch $k$ viele Knoten überdecken könnte.
			Mit $k$ Knoten mit $grad(v)=n$ kann man maximal $(n)\cdot k$ Kanten überdecken.\n
			Bildet $D$ eine $|D|$-Clique in $G$, so ist $k=|D|-2$ und bildet eine Ausnahme, da man mit $k$ vielen Knoten zwar rechnerisch genug Kanten abdecken könnte, jedoch zwei Knoten in der $|D|$-Clique übrig bleiben, die gemeinsam wieder eine Kante hätten.
			Diese Kante würde nicht abgedeckt werden.\n
			Wenn also $|D|>k$ gilt, gibt es keine Knotenüberdeckung in $G$.
		\subsubsection{}
			\textbf{Hinrichtung}\\
			Angenommen $G$ hat eine Knotenüberdeckung mit $k$ Knoten.\\
			In $G$ und $G'$ werden nun alle Kanten getroffen, wobei es Kanten von $G$ zu seinem Untergraphen $G'$ gibt. Nimmt man nun aus $G$ eben die $|D|$ vielen Knoten heraus durch die man $G'$ erhält, so gibt es keine Kanten mehr, die von $G$ nach $G'$ führen. Kanten, die nur innerhalb von $G'$ existieren bleiben jedoch bestehen.\\
			Somit hat $G$ eine $k'$ Knotenüberdeckung, wenn es eine $k$ Knotenüberdeckung in $G$ gibt.\n
			\textbf{Rückrichtung}\\
			Angenommen $G'$ hat eine Knotenüberdeckung mit $k'$ Knoten.\\
			Fügt man die in $D$ enthaltenen Knoten zu $G'$ hinzu, so hat $G$ genau $k$ viele Knoten, da $k'=k-|D| \Leftrightarrow k=k'+|D|$ gilt.
			Alle hinzugefügten Knoten nimmt man in die Knotenüberdeckung von $G$ mit auf, somit sind alle zu diesen ''neuen'' Knoten inzidenten Kanten auch in der Knotenüberdeckung enthalten und es gibt keine Kanten, die nicht getroffen werden.\\
			Somit hat $G$ eine Knotenüberdeckung mit $k$ Knoten, wenn $G'$ eine mit $k'$ vielen Knoten hat.
		\subsubsection{}
			Im folgenden wird angenommen, dass $G'$ genau $k'\cdot(k+1)$ viele Knoten und eine $k'$ Knotenüberdeckung besitzt. 
			Hat $G'$ weniger Knoten, so kann man welche Hinzufügen, ohne, dass sich etwas ändert.\n
			Behauptung:\\
			Die Knoten der $k'$ Knotenüberdeckung können keine weiteren Kanten aufnehmen.\n
			Beweis:\\
			Jeder der $k'$ vielen Knoten in der Überdeckung hat genau $k$ viele zu ihm inzidente Kanten in $G'$.
			Dadurch gibt es neben diesen Knoten noch $k'\cdot k$ weitere Knoten, die mit den $k'$ vielen Knoten der Überdeckung die insgesamt $k\cdot k'+k'=k'\cdot (k+1)$ viele Knoten in $G'$ ergeben.\n
			Angenommen einer der $k'$ vielen Knoten hat mehr als $k$ Kanten, dann wäre er in $D$ und nicht mehr in $G'$, wodurch $G'$ keine $k'\cdot(k+1)$ Knoten mehr hätte.
			Angenommen einer der $k'$ vielen Knoten hat weniger als $k$ viele Kanten, dann müsste ein anderer Knoten $v$ aus der Überdeckung $k+1$ viele Kanten besitzen, damit $G'$ weiterhin $k'\cdot (k+1)$ Knoten und eine $k'$ Knotenüberdeckung besitzt.
			Dadurch wäre aber $v$ in $D$ und nicht mehr in $G'$, wodurch $G'$ keine $k'\cdot(k+1)$ vielen Knoten mehr hätte.\\
			Es hat also jeder Knoten der $k'$ Knotenüberdeckung in $G'$ genau $k$ viele zu ihm inzidente Kanten.\n
			Behauptung:\\
			Enthält $G'$ mindestens $k'\cdot(k+1)+1$ Knoten, so gibt es keine Überdeckung in $G$ und $G'$.\n
			Beweis:\\
			Angenommen $G'$ enthält $k'\cdot(k+1)+n$ Knoten, dann folgt aus obigen, dass die zusätzlichen Knoten $v_1,\dots,v_n$ jeweils nicht mit einem Knoten $u$ aus der Knotenüberdeckung in $G'$ verbunden sein dürfen (da $u$ sonst $k+1$ inzidente Kanten besäße und nicht in $G'$ wäre, s.o.).\\
			Daher muss er mit einem Knoten $w$ verbunden sein, der nicht in der Knotenüberdeckung aber noch in $G'$  ist.
			Folglich gibt es eine Kante $(w, v_i)$ mit $1\leq i\leq n$, welche weder durch die Knotenüberdeckung in $G'$, noch durch die in $G$ getroffen wird.\n
			In $G'$ kann $(w,v_i)$ nicht getroffen werden, da sich dann die Anzahl der Knoten in der Überdeckung ändern würde, die jedoch durch $k'=k-|D|$ fest gegeben ist.\\
			In $G$ kann $(w, v_i)$ aber auch nicht getroffen werden, da jeder zusätzliche Knoten $v_i$ nicht mit einem Knoten der Überdeckung in $G$, sondern nur mit ''unbeteiligten'' Knoten verbunden ist. Nimmt man also jeden Knoten $v_i$ mit in die Überdeckung aus $G$ mit auf, erhöht sich dadurch die Größe der Überdeckung für jeden zusätzlichen Knoten, womit die minimale Überdeckung in $G$ die Größe $k+n$ hätte.\n
			Dadurch ist es nicht möglich, dass $G'$ mehr als $k'\cdot(k+1)$ Knoten enthält und $G$ eine Überdeckung mit der Größe $\leq k$ besitzt.\qed
		\subsubsection{}
			Man kann beim Einlesen von $E(G)$ für jeden Knoten dessen Grad speichern indem man für jede Kante den Grad der beiden Knoten jeweils inkrementiert.
			Dies ist mit indizierten Knoten in $\Ovon{1}$ machbar, ansonsten in $\Ovon{|V(G)|}$.\\
			Mit dem gespeicherten Grad für jeden Knoten kann man durch einmaliges iterieren über die Knotenmenge $V(G)$ die Menge $D$ erzeugen. Das entspricht einer Laufzeit von $\Ovon{|V(G)|}$.\\
			Einlesen, speichern und $D$ erzeugen ist also in $\Ovon{|V(G)|}$ möglich.\n
			Auch das löschen von $D$ aus $G$ geht schnell, denn man kann in $\Ovon{|V(G)|}$ die Knoten und in $\Ovon{|E(G)|}$ die Kanten löschen.
			Speichert man alle Mengenkardinalitäten zwischen (also von $G, V(G), D$), was kaum Zeit kostet bei jeder Änderung die Kardinalität zu aktualisieren, so kann man in konstanter Zeit prüfen, ob $|V(G')|>k'\cdot(k+1)$ gilt.
\end{document}