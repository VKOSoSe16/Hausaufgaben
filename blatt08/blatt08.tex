\documentclass[a4paper]{article}
\usepackage[ngerman]{babel}  
\usepackage[utf8]{inputenc}
\usepackage{amsmath}
\usepackage{amsfonts}
\usepackage{amssymb}
\usepackage{nicefrac}
\usepackage{nameref}

\usepackage{gail}
\usepackage{dadp}
\usepackage{makrocol}

% Wenn das installieren in /usr/share/texlive/texmf-dist/tex/generic/hauke96/ schief ging, dann einfach
% sh getpackages.sh locally 
% ausführen. Das aber NUR wenn es wirklich nicht geht
%\usepackage{gail}
%\usepackage{dadp}
%\usepackage{makrocol}

\setfirstauthor{Walter Stieben}
\setfirstauthorID{4stieben@inf}
\setsecondauthor{Tim Reipschläger}
\setsecondauthorID{4reipsch@inf}
\setthirdauthor{Louis Kobras}
\setthirdauthorID{4kobras@inf}
\setfourthauthor{Hauke Stieler}
\setfourthauthorID{4stieler@inf}
\settitle{Lösungsstrategien für NP-schwere Probleme der
Kombinatorischen Optimierung}
\setsheetnumber{8}
\setstartdate{2016}{04}{11}
\setdatefreq{7}
\setinterruptions{2}
\setsectionstyletasksalphnum{}

\begin{document}
	\maketitle
	\section{}
		Zunächst sei bemerkt, dass $c(T)\leq c(H^*)$ gilt, alle Kanten in $T$ haben weniger oder gleich viele Kosten wie die aus $H^*$.\n
		Beweis: $T$ ist ein \textit{minimaler} Spannbaum, man kann also keine Kanten weglassen und trotzdem einen zusammenhängenden Graphen haben und die vorhandenen Kanten sind diejenigen mit minimalem Gewicht womit $c(T)\not> c(H^*)$ gilt.\\
		Vor dem Bilden von $T^+$ gilt für $T$ die Ungleichung $c(T)<c(H^*)$.
		Da $T$ ein Baum ist und keinen Zyklus bildet, enthält $T$ keine Tour für das $\Delta$--TSP, somit kann $c(T)=c(H^*)$ nie gelten und es gilt $c(T)<c(H^*)$.\\
		Durch die Hinzunahme von $M$ gibt es einen Zyklus für das $\Delta$--TSP, somit ist $c(T^+)=c(H^*)$, zudem ist $c(T^+)=c(H)=c(T)+c(M)$.
		aus $(*)$ wissen wir, dass $c(M)\leq \frac{1}{2}\cdot c(H^*)$ gilt.
		Da alle Kanten aus $T^+$ in $H^*$ enthalten sind gilt für $T^+$:
		\[c(T^+)=c(H)=c(T)+c(M)\leq c(H^*)+\frac{1}{2}\cdot c(H^*)=\frac{3}{2}\cdot c(H^*)\]
		Insgesamt gilt also $c(H)\leq \frac{3}{2}\cdot c(H^*)$.\qed
	\section{}
\end{document}
