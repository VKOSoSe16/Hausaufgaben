\documentclass[a4paper]{article}
\usepackage[ngerman]{babel}  
\usepackage[utf8]{inputenc}
\usepackage{amsmath}
\usepackage{amsfonts}
\usepackage{amssymb}
\usepackage{nicefrac}
\usepackage{nameref}

\usepackage{gail}
\usepackage{dadp}
\usepackage{makrocol}

% Wenn das installieren in /usr/share/texlive/texmf-dist/tex/generic/hauke96/ schief ging, dann einfach
% sh getpackages.sh locally 
% ausführen. Das aber NUR wenn es wirklich nicht geht
%\usepackage{gail}
%\usepackage{dadp}
%\usepackage{makrocol}

\setfirstauthor{Walter Stieben}
\setfirstauthorID{4stieben@inf}
\setsecondauthor{Tim Reipschläger}
\setsecondauthorID{4reipsch@inf}
\setthirdauthor{Louis Kobras}
\setthirdauthorID{4kobras@inf}
\setfourthauthor{Hauke Stieler}
\setfourthauthorID{4stieler@inf}
\settitle{Lösungsstrategien für NP-schwere Probleme der
Kombinatorischen Optimierung}
\setsheetnumber{8}
\setstartdate{2016}{04}{11}
\setdatefreq{7}
\setinterruptions{2}
\setsectionstyletasksalphnum{}

\begin{document}
	\maketitle
	\section{}
		Zunächst sei bemerkt, dass $c(T)\leq c(H^*)$ gilt, alle Kanten in $T$ haben weniger oder gleich viele Kosten wie die aus $H^*$.\n
		Beweis: $T$ ist ein \textit{minimaler} Spannbaum, man kann also keine Kanten weglassen und trotzdem einen zusammenhängenden Graphen haben und die vorhandenen Kanten sind diejenigen mit minimalem Gewicht womit $c(T)\not> c(H^*)$ gilt.\\
		Vor dem Bilden von $T^+$ gilt für $T$ die Ungleichung $c(T)<c(H^*)$.
		Da $T$ ein Baum ist und keinen Zyklus bildet, enthält $T$ keine Tour für das $\Delta$--TSP, somit kann $c(T)=c(H^*)$ nie gelten und es gilt $c(T)<c(H^*)$.\\
		Durch die Hinzunahme von $M$ gibt es einen Zyklus für das $\Delta$--TSP, somit ist $c(T^+)=c(H^*)$, zudem ist $c(T^+)=c(H)=c(T)+c(M)$.
		aus $(*)$ wissen wir, dass $c(M)\leq \frac{1}{2}\cdot c(H^*)$ gilt.
		Da alle Kanten aus $T^+$ in $H^*$ enthalten sind gilt für $T^+$:
		\[c(T^+)=c(H)=c(T)+c(M)\leq c(H^*)+\frac{1}{2}\cdot c(H^*)=\frac{3}{2}\cdot c(H^*)\]
		Insgesamt gilt also $c(H)\leq \frac{3}{2}\cdot c(H^*)$.\qed
	\section{}
		\begin{pseudocode}{Approx--3D--Matching}{$T,M$}
			\Repeat
				\State Nehme erstes Tripel $t\in T$ in $M$ auf
				\For{alle $t_i\in T$}
					\If{$t\cap t_i\neq\emptyset$}
						\State lösche $t$ aus $T$
					\EndIf
				\EndFor
			\Until{$T=\emptyset$}
		\end{pseudocode}
		Um zu zeigen, dass für das mit diesem Algorithmus gefundene $M$ die Ungleichung $|M|\geq\frac{1}{3}\cdot |M^*|$ gilt machen wir folgende Annahme:\n
		Wir betrachten 4 Tripel aus $T$ und nehmen an, dass 3 dieser Tripel Elemente von $M^*$ sind.
		Wir nehmen weiterhin an, dass das übrig gebliebene Tripel jeweils ein Element mit jedem der 3 Tripel aus $M^*$ gemeinsam hat.\\
		Nehmen wir nun das Tripel in $M$ auf, das nicht in $M^*$ ist, so streichen wir genau die 3 Tripel, die Element $M^*$ sind von der optimalen Lösung.
		Selbst wenn es mehrere optimale Lösungen gibt (also mehrere $M^*$ mit gleichen Betrag), streichen wir aus jeder dieser optimalen Lösungen genau 3.\n
		Wir können offensichtlich nur maximal 3 Tripel jeder beliebigen optimalen Lösung $M^*$ streichen, da die 3 Elemente aus unserem schlecht gewählten Tripel maximal in 3 unterschiedlichen Tripeln von $M^*$ vorkommen können.
		Das liegt daran, dass sich die Tripel aus $M^*$ nicht überschneiden dürfen, da es sonst keine gültige Lösung wäre und somit kann ein Element unseres schlecht gewählten Tripels immer nur genau in einem Tripel aus $M^*$ Vorkommen.
		Wenn wir für unsere Annahme nun $|M|$ und $|M^*|$ vergleichen, dann haben wir in $M$ ein Tripel und in $M^*$ 3 Tripel. Demnach ist $|M|= \frac{1}{3}\cdot |M^*|$ und wir sind noch in der Grenze.\n
		Der beschriebene Fall ist offensichtlich der worst-case, denn wenn unser gewähltes Tripel nur 2 Tripel aus $M^*$ schneidet, erhalten wir $|M|=1$, $|M^*|=2$ und somit $|M|=\frac{1}{2}\cdot |M^*|$, womit wir noch in der Grenze ist. Schneidet Das Schlecht gewählte Tripel nur ein Element aus $M^*$, muss das gewählte Tripel in einer anderen optimalen Lösung $M_{2}^*$ sein, da man die 2 Tripel offensichtlich austauschen kann. Allerdings würde dieser Fall der Annahme widersprechen.\n
		Somit findet unser Algorithmus immer Mengen $M$, für die gilt $|M|\geq \frac{1}{3}|M^*|$.
\end{document}
