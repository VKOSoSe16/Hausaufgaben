\documentclass[a4paper]{article}
\usepackage{amsmath}
\usepackage{amsfonts}
\usepackage{amssymb}
\usepackage[ngerman]{babel}  
\usepackage[utf8]{inputenc}

\usepackage{../gail}
\usepackage{../dadp}

\setfirstauthor{Walter Stieben}
\setfirstauthorID{4stieben@inf}
\setsecondauthor{Tim Reipschläger}
\setsecondauthorID{4reipsch@inf}
\setthirdauthor{Louis Kobras}
\setthirdauthorID{4kobras@inf}
\setfourthauthor{Hauke Stieler}
\setfourthauthorID{4stieler@inf}
\settitle{Lösungsstrategien für NP-schwere Probleme der
Kombinatorischen Optimierung}
\setsheetnumber{2}
\setstartdate{2016}{04}{11}
\setdatefreq{7}

\begin{document}
	\maketitle
	\section{}
		\subsection{}
		\subsection{Spezialfall $k=2$}
			Beim Spezialfall für $k=2$ kann man tatsächlich einen Algorithmus angeben, dessen Laufzeit polynomiell in der Eingabe ist.
		\alglanguage{pseudocode}
		\begin{breakablealgorithm}
			\caption{reserve}
			\begin{algorithmic}[1]
				\Procedure{reserve}{P, R, k}
					\ForAll{p$\in$P}
						\State R$_{rest}$ = weiseResourcenZu(p, R) // gibt alle noch nicht zugewiesenen Ressourcen zurück
						\State P = P $-$ p
						\ForAll{q$\in$P}
							\State b = istVerteilungMöglich(q, R$_{rest}$)
							\If{b == true}
								\Return true
							\EndIf
						\EndFor
					\EndFor
					\Return false
				\EndProcedure
			\end{algorithmic}
		\end{breakablealgorithm}
		Die Laufzeit beträgt dabei $\Ovon{|P|\cdot n+|P|^2\cdot n}$, wobei $n=|R|$ ist.\n
		Man Geht in der äußeren Schleife alle $p\in P$ durch , also $|P|$ mal. Dort verteilt man zunächst maximal $n$ viele Ressourcen, ergo $n$ viele Schritte. Nun wird $p$ aus $P$ entfernt, was mit einer schlauen Implementierung (z.B. als Liste) in konstanter Zeit machbar ist.\\
		Danach beginnt die innere Schleife, welche alle $q\in P$ durch geht und somit $|P|-1$ mal durchläuft. Dort wird dann geprüft ob eine weitere Verteilung der restlichen Ressourcen auf $q$ möglich ist. Auch da benötigt man maximal $n$ viele Schritte.\\
		Man erhält also eine Laufzeit von $\Ovon{|P|\cdot(n+|P|\cdot n)}$, was mit $\Ovon{|P|\cdot n+|P|^2\cdot n}$ äquivalent ist (hier erkennt man aber schön das Polynom).
		\subsection{}
		\subsection{}
	\section{}
		\subsection{}
		\subsection{}
\end{document}