\documentclass[a4paper]{article}
\usepackage[ngerman]{babel}  
\usepackage[utf8]{inputenc}
\usepackage{amsmath}
\usepackage{amsfonts}
\usepackage{amssymb}
\usepackage{nicefrac}
\usepackage{nameref}

\usepackage{gail}
\usepackage{dadp}
\usepackage{makrocol}

% Wenn das installieren in /usr/share/texlive/texmf-dist/tex/generic/hauke96/ schief ging, dann einfach
% sh getpackages.sh locally 
% ausführen. Das aber NUR wenn es wirklich nicht geht
%\usepackage{gail}
%\usepackage{dadp}
%\usepackage{makrocol}

\setfirstauthor{Walter Stieben}
\setfirstauthorID{4stieben@inf}
\setsecondauthor{Tim Reipschläger}
\setsecondauthorID{4reipsch@inf}
\setthirdauthor{Louis Kobras}
\setthirdauthorID{4kobras@inf}
\setfourthauthor{Hauke Stieler}
\setfourthauthorID{4stieler@inf}
\settitle{Lösungsstrategien für NP-schwere Probleme der
Kombinatorischen Optimierung}
\setsheetnumber{7}
\setstartdate{2016}{04}{11}
\setdatefreq{7}
\setinterruptions{1}
\setsectionstyletasksalphnum{}

\begin{document}
	\maketitle
	\section{}
		\begin{pseudocode}{ApproxWightedHittingSet}{$A, B$}
			\Procedure{ApproxWightedHittingSet}{$A, B$}
				\While{$B\neq \emptyset$}
					\State $a$ := \Null \hfill // the element with the best quality
					\State $q_{min}:=\infty$
					\ForAll{$a_i\in A$}
						\State $n := \text{amountOfAppearances}(a_i)$
						\State $quality$ := $weight(a_i) / n$
						\If{$quality<q_{min}$}
							\State $q_{min}\gets quality$
							\State $a\gets a_i$
						\EndIf
					\EndFor
					\State $H\gets H\cup \{a\}$
					\State $B\gets B\setminus\text{allSetsHitBy}(a)$
					\State $A\gets A\setminus a$
				\EndWhile
			\EndProcedure
		\end{pseudocode}
		\subsubsection*{Beschreibung}
		Zu Beginn jedes Schleifendurchgangs (Zeile 3-15) wird das Element aus $A$ bestimmt, welches die beste Qualität hat.
		Die Qualität beschreibt wie viel Gewicht pro getroffenem $B_j\in B$ aufgenommen wird.
		Je weniger Gewicht aufgenommen wird, desto besser das Resultat. Dadurch hat ein kleinerer Wert eine höhere (bessere) Qualität.\n
		Bestimmt wird das Element $a$ mit der besten Qualität in Zeile 5-12 in der alle verbleibenden $a_i\in A$ durchgegangen werden.
		Die Funktion \texttt{amountOfAppearances} berechnet dabei die Anzahl der getroffenen Mengen $B_j\in B$ durch das übergebene Element $a_i$.\n
		Am Ende (Zeile 13-15) wird das gefundene Optimum für diesen Durchlauf ($a$) in das Hitting Set $H$ aufgenommen.
		Zudem wird aus $B$ jede Menge entfernt, die durch $a$ getroffen wurde.
		Auch wird $a$ aus $A$ entfernt, sodass nun ein neues Optimum berechnet werden kann.
		\subsubsection*{Terminierung}
		Der angegebene Algorithmus terminiert, da in jedem $B_i\in B$ nur Elemente aus $A$ vorkommen. Da im worst-case jedes $A$ einmal die beste Qualität hat, wird jedes $B_i$ getroffen. Elemente mit gleicher Qualität werden nacheinander behandelt, sodass es auch dort keine Terminierungsprobleme geben kann.
		\subsubsection*{Laufzeitanalyse}
		Auch wenn die Bedingung der \texttt{while}-Schleife $B\neq\emptyset$ lautet, so wird diese maximal $|A|$ mal ausgeführt, da bei jedem Durchlauf $A$ um ein Element verkleinert wird.
		Ist $A=\emptyset$, so ist auch $B=\emptyset$ (s. Abschnitt Terminierung oben).\n
		Die innere \texttt{for}-Schleife wird ebenfalls $|A|$ mal ausgeführt.
		In ihr wird \texttt{amountOfAppearances} aufgerufen, was die Anzahl der ''Hits'' ausgibt.
		Dabei wird in einer intuitiven Implementation jede Menge in $B$ in jedes Element in dieser Menge durchgegangen.
		Jede Menge in $B$ kann dabei maximal $|A|$ viele Elemente beherbergen, wodurch sich eine Laufzeit ergibt, die in $\Ovon{|B|\cdot |A|}$ ist.
		Alle anderen Schritte haben eine konstante Laufzeit.\n
		Am Ende (Zeile 13-15) wird $a$ in $H$ aufgenommen, was in konstanter Zeit machbar ist und $a$ aus $A$ gelöscht, was in linearer Zeit machbar ist (sofern man bei beiden von einer verketteten Liste ausgeht).\\
		Die Funktion \texttt{allSetsHitBy}, welche Aufgerufen wird funktioniert in einer intuitiven Implementation genauso wie \texttt{amountOfAppearances}, nur wird hier ein anderes Ergebnis zurückgegeben. 
		Die Laufzeit von \texttt{allSetsHitBy} ist somit ebenfalls in $\Ovon{|B|\cdot |A|}$.\n
		Insgesamt ergibt sich also eine in der Eingabe polynomielle Laufzeit in \[\Ovon{|A|\cdot (|A|\cdot (|B|\cdot |A|) + (|B|\cdot |A|))}=\Ovon{|A|^3\cdot |B|+|A|^2\cdot |B|}.\]
		
		\subsubsection*{Beweis: \texttt{ApproxWightedHittingSet} ist ein $b$-Approximationsalgorithmus}
		
	\section{}
\end{document}
