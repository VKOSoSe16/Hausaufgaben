\documentclass[a4paper]{article}
\usepackage[ngerman]{babel}  
\usepackage[utf8]{inputenc}
\usepackage{amsmath}
\usepackage{amsfonts}
\usepackage{amssymb}
\usepackage{nicefrac}

\usepackage{gail}
\usepackage{dadp}
\usepackage{makrocol}

% Wenn das installieren in /usr/share/texlive/texmf-dist/tex/generic/hauke96/ schief ging, dann einfach
% sh getpackages.sh locally 
% ausführen. Das aber NUR wenn es wirklich nicht geht
%\usepackage{gail}
%\usepackage{dadp}
%\usepackage{makrocol}

\usepackage{multicol}

\setfirstauthor{Walter Stieben}
\setfirstauthorID{4stieben@inf}
\setsecondauthor{Tim Reipschläger}
\setsecondauthorID{4reipsch@inf}
\setthirdauthor{Louis Kobras}
\setthirdauthorID{4kobras@inf}
\setfourthauthor{Hauke Stieler}
\setfourthauthorID{4stieler@inf}
\settitle{Lösungsstrategien für NP-schwere Probleme der
Kombinatorischen Optimierung}
\setsheetnumber{7}
\setstartdate{2016}{04}{11}
\setdatefreq{7}
\setinterruptions{1}
\setsectionstyletasksalphnum{}

\begin{document}
	\maketitle
	\section{}
%		\begin{pseudocode}{ApproxWightedHittingSet}{$A, B$}
%%			\State $r$ := 0
%			\State $H$ := $\emptyset$
%			\ForAll{$A_i\in A$}
%				\State $E$ := sortToList($A_i$)
%				\State $n_{min}$ := $\infty$
%				\State $h$ := \Null
%				\ForAll{$e_j\in E$}
%					\State $n$ := amountOfHits($e_j$)
%					\If{$n<n_{min}$}
%						\State $n_{min} \gets n$
%						\State $h\gets e_j$
%					\EndIf
%				\EndFor
%				\State $H \gets H\cup \{h\}$
%				\State $A\gets A\setminus\text{allSetsHitBy(h)}$
%			\EndFor
%		\end{pseudocode}
		\begin{pseudocode}{ApproxWightedHittingSet}{$A, B$}
			\State $R$ := relation from $element \rightarrow quality ~index$
			\ForAll{$a_i\in A$}
				\State $n$ := amountOfAppearances($a_i$)
				\State $quality$ := $weight(a_i) / n$
				\State addToRelation($R$, $a_i\rightarrow quality$)
			\EndFor
			\ForAll{$a_i \in R$ \textbf{with} $R(a_k)\leq R(a_{k+1})$ \textbf{and} $B \neq \emptyset$}
				\State $H\gets H\cup \{a_i\}$
				\State $B\gets B\setminus\{$allSetsHitBy$(a_i)\}$
			\EndFor
		\end{pseudocode}
		\subsubsection*{Beschreibung}
		$R$ ist eine Relation, die jedes $a_i\in A$ auf einen Qualitätsindex abbildet.
		Dieser Index ist einfach das Gewicht $\frac{weight(a_i)}{n}$, wobei $weight(a_i)$ das Gewicht von $a_i$ ist und $n$ die Anzahl der Vorkommen angibt.\n
		Die zweite \texttt{for}-Schleife (Zeile 8-11) geht alle $a_i$ durch, wobei bei dem $a_i$ mit den geringsten Gewicht begonnen wird, sodass $R(a_k)\leq R(a_{k+1})$ gilt.
		Zudem bricht die Schleife ab, sobald kein $B_j\in B$ mehr existiert ($A$ muss dabei nicht leer sein), sprich sobald $B=\emptyset$ gilt.\n
		In der Schleife wird nun jedes $a_i$ in $H$ aufgenommen und jedes getroffene $B_j$ aus $B$ entfernt, sodass man keine Elemente doppelt aufgenommen werden, die evtl. gar keine neuen $B_j$ treffen.
	\section{}
\end{document}
