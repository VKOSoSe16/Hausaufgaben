\documentclass[a4paper]{article}
\usepackage[ngerman]{babel}  
\usepackage[utf8]{inputenc}
\usepackage{amsmath}
\usepackage{amsfonts}
\usepackage{amssymb}
\usepackage{nicefrac}

\usepackage{../gail}
\usepackage{../dadp}
\usepackage{../makrocol}

\newcommand{\explore}{\texttt{EXPLORE$(\Phi, d)$}\xspace}

\setfirstauthor{Walter Stieben}
\setfirstauthorID{4stieben@inf}
\setsecondauthor{Tim Reipschläger}
\setsecondauthorID{4reipsch@inf}
\setthirdauthor{Louis Kobras}
\setthirdauthorID{4kobras@inf}
\setfourthauthor{Hauke Stieler}
\setfourthauthorID{4stieler@inf}
\settitle{Lösungsstrategien für NP-schwere Probleme der
Kombinatorischen Optimierung}
\setsheetnumber{5}
\setstartdate{2016}{04}{11}
\setdatefreq{7}
\setinterruptions{1}
\setsectionstyletasksalphnum{}

\begin{document}
	\maketitle
	\section{}
	\section{}
	\subsubsection{}
	% TODO Hier restliche Aufgabe einfügen!
	
	\subsubsection*{Laufzeitanalyse}
	Die Laufzeit des von \explore lässt sich mittels Rekurrenzgleichung und Substitutionsmethode bestimmen.\n
	Zunächst die Rekurrenzgleichung in Abhängigkeit von $d$ und $n$.
	Die Menge $C$ taucht nicht auf, da sie nicht Teil der Eingabe von \explore ist. Dem $n$ wäre also noch ein linearer Faktor hinzu zu zählen, was jedoch wenig von Bedeutung ist.
	\[
		T(d, n)=
		\begin{cases}
			n					& , \text{für } d=0\\
			n\cdot 3\cdot T(d-1)		& , \text{sonst}
		\end{cases}
	\]
	Mittels der Sibstitutionsmethode ergibt sich folgende Kette:
	\begin{align*}
		T(d, n)	&=		n+3		\cdot 						T(d-1)			\\
				&=		n+3		\cdot 	(	n+3		\cdot 	T(d-2))			\\
				&=		n+3			 		n+9		\cdot 	T(d-2)			\\
				&=	4 	n+9		\cdot 						T(d-2)			\\
				&=	4 	n+9		\cdot	(	n+3		\cdot 	T(d-3))			\\
				&=	4	n+9					n+27	\cdot	T(d-3)			\\
				&=	13	n+27	\cdot						T(d-3)			\\
				&=	13	n+27	\cdot	(	n+3		\cdot	T(d-4))			\\
				&=	13	n+27				n+81	\cdot	T(d-4)			\\
				&=	40	n+81	\cdot						T(d-4)	\tag{1}	\\
				&\makebox[\widthof{${}={}$}][c]{\vdots}					\\
				&=\frac{3^k-1}{2}+3^k\cdot T(d-k)
	\end{align*}
	Nach drei Substitutionen kann man erkennen, der Vorfaktor von $T$ stets $3^k$ ist $(3, 9, 27, 81, \dots)$ und der Faktor vom $n$ stets $\frac{3^k-1}{2}$ sein muss $(1, 4, 13, 40, \dots)$.
	\begin{note}
	Beweise zu den Gleichungen ersparen wir uns hier der einfachheitshalber.
	\end{note}
	Der Abbruch des Algorithmus' geschieht bei $k=d$.
	Dadurch ist $T(d-k)=T(d-d)=T(0)=n$ und wir erhalten die finale Laufzeit folgender Größenordnung:
	\[
		\Ovon{\frac{3^d-1}{2}+3^d\cdot n}
	\]
	\subsubsection{}
	Wenn man $d=\frac{n}{2}$ wählt, so erhält und in die Laufzeit aus 2.a) einsetzt, erhält man folgende neue Laufzeit:
	\begin{align*}
			\frac{3^{\nicefrac{n}{2}}-1}{2}+3^{\nicefrac{n}{2}}\cdot n
		&=	3^{\nicefrac{n}{2}}\cdot 0,5- \left(\frac{1}{2}\right)+3^{\nicefrac{n}{2}}\cdot n
			\tag{Die $\mathcal{O}$-Notation ignoriert Konstanten}\\
		&=	3^{\nicefrac{n}{2}}+3^{\nicefrac{n}{2}}\cdot n\\
		&=	\left(\sqrt{3}\right)^n+\left(\sqrt{3}\right)^n\cdot n\\
		&=	(n+1)\cdot \left(\sqrt{3}\right)^n
	\end{align*}
	\vspace{-2ex}
	\begin{note}
	Die Notation $\nicefrac{n}{2}$ statt $\frac{n}{2}$ wird zwar von Mathematikern und Theoretikern oftmals abgelehnt, wird hier jedoch zugunsten der Lesbarkeit verwendet.
	\end{note}
	Somit ist die Laufzeit von \explore mit $d=\frac{n}{2}$ in $\Ovon{(n+1)\cdot \left(\sqrt{3}\right)^n}$.\n
	Um jedoch alle möglichen Belegungen durch zu gehen bedarf es $\dfrac{2^n}{\frac{n}{2}}=\frac{2^{n+1}}{n}$ Aufrufe. Jeder Aufruf von \explore geht $d=\frac{n}{2}$ Belegungen durch, wodurch man nicht volle $2^n$ mögliche Belegungen benötigt.\n
	Die Anzahl multipliziert mit der Anzahl an Schritten aus vorheriger Rechnung ergibt die Gesamtlaufzeit von $\Ovon{\frac{2^{n+1}}{n}\cdot (n+1)\cdot \left(\sqrt{3}\right)^n}$.\n
	Somit haben wir eine Laufzeit der Form $\Ovon{p(n)\cdot \left(\sqrt{3}\right)^n}$.
\end{document}