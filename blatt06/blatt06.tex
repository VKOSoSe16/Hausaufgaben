\documentclass[a4paper]{article}
\usepackage[ngerman]{babel}  
\usepackage[utf8]{inputenc}
\usepackage{amsmath}
\usepackage{amsfonts}
\usepackage{amssymb}
\usepackage{nicefrac}

\usepackage{../gail}
\usepackage{../dadp}
\usepackage{../makrocol}

\setfirstauthor{Walter Stieben}
\setfirstauthorID{4stieben@inf}
\setsecondauthor{Tim Reipschläger}
\setsecondauthorID{4reipsch@inf}
\setthirdauthor{Louis Kobras}
\setthirdauthorID{4kobras@inf}
\setfourthauthor{Hauke Stieler}
\setfourthauthorID{4stieler@inf}
\settitle{Lösungsstrategien für NP-schwere Probleme der
Kombinatorischen Optimierung}
\setsheetnumber{6}
\setstartdate{2016}{04}{11}
\setdatefreq{7}
\setinterruptions{1}
\setsectionstyletasksalphnum{}

\begin{document}
	\maketitle
	\section{}
		\subsubsection{}
		Zu zeigen ist, dass der angegebene Algorithmus kein 2-Approximationsalgorithmus ist. Zeigen kann man das mit einem Gegenbeispiel:\n
		Sei $A=\{1, 2, 8\}$ und $B=10$. Der Algorithmus findet nun folgende Mengen:
		\begin{center}
			\begin{tabular}{c|c}
				Index $i$ & Gefundene Menge $S$\\\hline
				1 & $\{1\}$\\
				2 & $\{1, 2\}$\\
				3 & $\{1, 2\}$\\
			\end{tabular}
		\end{center}
		Der Algorithmus nimmt keine Zahlen mehr ab dem Index auf, da dann die Bedingung $\sum\limits_{a_i\in S} a_i\leq B$ nicht mehr gelten würde, da $1+2+8=11>10$ gilt.\n
		Das Ergebnis erfüllt somit nicht die Bedingung eines $\rho$-Approqimationsalgorithmus für Maximierungsprobleme $L^*/L_A\leq \rho$. Stattdessen gilt für das Ergebnis $L_A=3$, die totale Summe $L^*=B=10$ und $\rho=2$ die Gleichung $L^*/L_A=10/3=\overline{3,3}\not\leq\rho$.\n
		Damit ist der angegebene Algorithmus kein 2-Approximationsalgorithmus.\qed
		\subsubsection{}
	\section{}
\end{document}