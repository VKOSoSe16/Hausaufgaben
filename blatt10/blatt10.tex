\documentclass[a4paper]{article}
\usepackage[ngerman]{babel}  
\usepackage[utf8]{inputenc}
\usepackage{amsmath}
\usepackage{amsfonts}
\usepackage{amssymb}
\usepackage{nicefrac}
\usepackage{nameref}

\usepackage{gail}
\usepackage{dadp}
\usepackage{makrocol}

\usepackage{tikz}

% Wenn das installieren in /usr/share/texlive/texmf-dist/tex/generic/hauke96/ schief ging, dann einfach
% sh getpackages.sh locally 
% ausführen. Das aber NUR wenn es wirklich nicht geht
%\usepackage{gail}
%\usepackage{dadp}
%\usepackage{makrocol}

\setfirstauthor{Walter Stieben}
\setfirstauthorID{4stieben@inf}
\setsecondauthor{Tim Reipschläger}
\setsecondauthorID{4reipsch@inf}
\setthirdauthor{Louis Kobras}
\setthirdauthorID{4kobras@inf}
\setfourthauthor{Hauke Stieler}
\setfourthauthorID{4stieler@inf}
\settitle{Lösungsstrategien für NP-schwere Probleme der
Kombinatorischen Optimierung}
\setsheetnumber{10}
\setstartdate{2016}{04}{11}
\setdatefreq{7}
\setinterruptions{2}
\setsectionstyletasksalphnum{}

\begin{document}
	\maketitle
	\section{}
	Wir betrachten die folgenden Zufallsvariablen:
	\begin{itemize}
	\item $X:$ Die Anzahl der Stimmen für D.
	\item $X_1:$ Die Anzahl der Stimmen der Wähler von D, die D gewählt haben
	\item $X_2:$ Die Anzahl der Stimmen der Wähler von R, die D gewählt haben
	\item $X_3:$ Die Anzahl der Stimmen der Wähler von D, die R gewählt haben
	\end{itemize}
Es gilt die folgende Beziehung:
\[
	X = X_1 + X_2 - X_3
\] und damit auch \[
	E[X] = E[X_1 + X_2 - X_3]
\]. Aufgrund der linearität des Erwartungswertes können wir dies als Summe
\[
	E[X] = E[X_1] + E[X_2] - E[X_3]
\] schreiben. Wir rechnen nun:
\begin{align*}
E[X_1] &= (1-\frac{1}{100}) \cdot 80000 = 79200 \\
E[X_2] &= \frac{1}{100} \cdot 20000 = 200 \\
E[X_3] &= \frac{1}{100} \cdot 80000 = 800
E[X] &= 79200 + 200 - 800 = 78600
\end{align*}
Der Erwartungswert von X ist damit 78600.

\section{}
\begin{enumerate}
\item Immer dann, wenn wir einen Prozess in S aufnehmen, legen wir gleichzeitig fest, dass alle mit dem gerade aufgenommenen Prozess in Konflikt stehenden Knoten nicht in S aufgenommen werden. Jede einzelne Selektion ist damit Konfliktfrei und damit muss die Auswahl aller Prozesse in S am Ende auch konfliktfrei sein.

\item Nicht bearbeitet.
\end{enumerate}

\end{document}
